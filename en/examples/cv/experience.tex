%-------------------------------------------------------------------------------
%	SECTION TITLE
%-------------------------------------------------------------------------------

\cvsection{Experience}

%-------------------------------------------------------------------------------
%	CONTENT
%-------------------------------------------------------------------------------
\begin{cventries}
%---------------------------------------------------------
  \cventry
    {Development and validation of a minimally invasive robot system for particle implantation brachytherapy} % Job title
    {National key R\&D Development} % Organization
    {Shandong University} % Location
    {Apr. 2020 - Now} % Date(s)
    {
      \begin{cvitems} % Description(s) of tasks/responsibilities
        \item {Undertook the Research topic "Intraoperative Visual Guidance of Dynamic Multi-modality Fusion" as a key member of the project team.}
        % \item {Aiming to meet the requirements of image registration and visualization during intraoprative guidance, 
        % this topic carries out research dynamic registration, real-time positioning and tracking of spatial targets.}
        \item {Developed several medical image deep learning-based registration algorithm applied for CT and multi-modality MR images.}
        \item {Implemented a navigation and execution robot system for surgical which performed image guided puncture operation. }
        \item {Actualized the project management such as system integration commissoning, project declaration cooperation and demand scheduling.}
      \end{cvitems}
    }
%---------------------------------------------------------
  \cventry
    {Published 4 conference papers and applied for 1 patent. } % Job title
    {$\lceil$ 1 $\rfloor$ Research on medical image registration based on deep learning} % Organization
    {} % Location
    {} % Date(s)
    {
      \textbf{Registration Methods for 4D CT of Lung Based on Temporal-Spatial Features.}
      \begin{itemize}[leftmargin=*]
        \item Developed a Temporal-Spatial Features fusion registration method to make up the absence of temporal characteristics in existing methods.
        \item Designed a novel temporal modeling module inserted into the bottleneck layer of UNet to process motion information .
        \item Proposed a Dual-stream registration Pipeline to fit the relationship between data and network.
        \item Achieved significant improvement results. Realated works published on BioCAS 2022 (in oral).
      \end{itemize}
      \textbf{Multimodal Medical Image Registration based on Multi Metric Fusion}
      \begin{itemize}[leftmargin=*]
        \item Developed a multi metric registrion method combined intra-modality similarity and structure code metric to optimize network in a more global perspective.
        \item Extracted structure information from multimodal images by image dsentanglement as multimodal similarity metric.
        \item Designed fusion strategy for dual-metric registration model optimization with both intra-modality similarity and structure code.
        \item Outperformed on translation-based and single metric registration methods. Realated works published on ISBI 2022.
      \end{itemize}
    }
%---------------------------------------------------------
  \cventry
    {Development of Robot System, navigation, and related software  } % Job title
    {$\lceil$ 2 $\rfloor$ Development of surgical robot system with visual guidance} % Organization
    {} % Location
    {} % Date(s)
    {
      \begin{cvitems} % Description(s) of tasks/responsibilities
        \item Build the surgical navigation and execution system which takes NDI as the scene camera, UR5 as the executive mechanism, and Qt to design GUI.
        \item Standardized the navigation system functions, completed the calibration process of image, scene camera and robot.
        \item Realized TCP/IP communication with equipment for the exchange of tracking and guidance data
        \item Designed GUI for the functions of surgical navigation, dynamic display, and robot execution control.
      \end{cvitems}
    }
%---------------------------------------------------------
\cventry
{Coordination among topics, project document writing and integration} % Job title
{$\lceil$ 3 $\rfloor$ Project process organization and management} % Organization
{} % Location
{} % Date(s)
{
  \begin{cvitems} % Description(s) of tasks/responsibilities
    \item Wrote and integrated project declaration and sci-tech report.
    \item Coordinated with other units of the project to integrate the puncture robot system.
    \item Familiar with the planning, navigation, execution and clinical operation of the puncture surgery robot.
  \end{cvitems}
}
%---------------------------------------------------------
  % \cventry
  %   {Researcher} % Job title
  %   {Undergraduate Research, Machine Learning Lab(Prof. Seungjin Choi)} % Organization
  %   {Pohang, S.Korea} % Location
  %   {Mar. 2016 - Exp. Jun. 2017} % Date(s)
  %   {
  %     \begin{cvitems} % Description(s) of tasks/responsibilities
  %       \item {Researched classification algorithms(SVM, CNN) to improve accuracy of human exercise recognition with wearable device.}
  %       \item {Developed two TIZEN applications to collect sample data set and to recognize user exercise on SAMSUNG Gear S.}
  %     \end{cvitems}
  %   }
%---------------------------------------------------------
\end{cventries}
