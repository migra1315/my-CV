\cvsection{项目经历}


%----------------------------------------------------------------
%	内容
%----------------------------------------------------------------
\begin{cventries}
%---------------------------------------------------------
  \cventry
    {课题“动态多模影像融合的术中可视化导引技术研究”, 项目组骨干成员}% 职位名称
    {国家重点研发计划“粒子植入近距离放疗微创机器人系统研制及验证”}  % 组织
    {山东大学} % 位置
    {2020年4月至今} % 日期(s)
    {      
      \begin{cvitems} % 任务/职责描述
        \item{\textbf{项目描述: }本项目面向穿刺手术过程中对影像动态配准以及术中导引作业中可视化引导需求, 
        开展基于动态多模态影像数据融合的动态配准与空间目标实时定位、跟踪技术的研究。}
        \item{\textbf{承担工作: }医学图像配准算法开发;手术导航系统搭建与机器人系统集成;项目管理工作}
      \end{cvitems}
    }
  \cventry
    {发表会议论文4篇, 申请专利1项。} % Job title
    {「科研」基于深度学习的医学图像配准算法研究} % Organization
    {} % Location
    {} % Date(s)
    {
      \begin{cvitems}
        \item \textbf{基于时空特征的肺部4D CT配准算法}
        \begin{itemize}[leftmargin=*,topsep=5 pt, partopsep=5 pt]
          \item[$\ast$] 提出一种时空特征融合配准算法, 以弥补现有配准策略中时序运动信息的缺位。
          \item[$\ast$] 设计Conv-LSTM构成的时序建模模块作为 UNet 的 bottleneck 层, 用以处理图像序列的运动特征。
          \item[$\ast$] 构建Dual-stream配准流平衡网络完成对周期运动数据的处理。
          \item[$\ast$] 在Dirlab数据集上较其他算法取得显著提升, 相关工作发表在 BioCAS 2022 (in oral)。
        \end{itemize}
        \item \textbf{多标准融合的多模态医学影像配准算法}
        \begin{itemize}[leftmargin=*,topsep=5 pt, partopsep=5 pt]
          \item[$\ast$] 提出了一种结合跨模态相似度量和图像结构编码度量的多标准配准策略, 以更全局的角度优化网络。
          \item[$\ast$] 通过非成对图像训练图像翻译与解纠缠网络提取 (模态无关) 结构编码, 作为多模态影像相似性度量。
          \item[$\ast$] 设计多标准融合策略, 结合跨模态相似度量和图像结构编码度量以优化网络。
          \item[$\ast$] 优于现有基于图像翻译方法及单标准方法的配准结果。相关工作发表在 ISBI 2022。
        \end{itemize}
      \end{cvitems}
    }
    
  \cventry
    {合作完成。主要负责系统搭建、导航模块开发和GUI设计工作}% 职位名称
    {「项目」手术机器人系统搭建及可视化界面开发}  % 组织
    {} % 位置
    {} % 日期(s)
    {      
      \begin{cvitems} % 任务/职责描述
        \item{以NDI作为场景相机, UR5作为执行机构, 使用Qt设计GUI搭建手术导航与执行系统。}
        \item{规范导航系统功能, 实现影像、场景相机以及执行机构标定流程。}
        \item{搭建相机、机械臂以及HoloLens间的TCP/IP通信, 完成追踪与引导数据的传输交流。}
        \item{设计GUI, 满足手术注册、规划显示、器官及穿刺执行机构动态展示以及执行控制功能。}
      \end{cvitems}
    }

    \cventry
    {负责课题间联系协调、项目文书撰写整合以及系统联调}% 职位名称
    {「项目管理」项目进程组织管理}  % 组织
    {} % 位置
    {} % 日期(s)
    {      
      \begin{cvitems} % 任务/职责描述
        \item{撰写, 协调并整合科技报告、项目申报书。}
        \item{负责与项目其他单位协调集成穿刺机器人系统, 主要负责与手术规划系统和穿刺执行系统的对接和协调。}
        \item{对穿刺手术机器人的规划、导航、执行以及临床操作均有一定的认识了解。}
      \end{cvitems}
    }
%---------------------------------------------------------
  % \cventry
  %   {学院优秀毕业设计} % Job title
  %   {基于互信息的医学图像配准算法研究,毕业设计} % Organization
  %   {中国矿业大学} % Location
  %   {2019年12月-2020年6月} % Date(s)
  %   {
  %     \begin{cvitems} % Description(s) of tasks/responsibilities
  %       \item {项目描述:本项目是毕业设计,取得学院优秀毕业设计。针对多模态医学图像配准任务对精度和速度的需求,设计基于互信息MI的多模态医学图像配准算法研究,
  %       设计图像形态学处理、多分辨率等策略提高配准速度与精度。}
  %       \item {项目职责:使用MATLAB实现对肺部医学图像的形态学预处理,完成肺部特征的提取与增强。基于C++,使用OpenVC设计以互信息为相似性度量的多分辨率医学图像配准算法,使用Powell优化算法驱动图像变形,在BrainWeb数据集上验证配准精度可达到亚像素级别。}
  %     \end{cvitems}
  %   }

  %  %--------------------------------------------------------- 
%   \cventry
%     {电子设计} % Job title
%     {基于WiFi的智能家居系统设计,项目组长} % Organization
%     {中国矿业大学} % Location
%     {2019年5月-2019年6月} % Date(s)
%     {
%       \begin{cvitems} % Description(s) of tasks/responsibilities
%         \item {项目描述:本项目是电子设计。设计实现手机端APP与设备端单片机通过WiFi实现通信的智能家居系统。}
%         \item {项目职责:使用Java实现Android端APP设计,完成对手机传感器等信息的采集、分析,通过WiFi与受控端建立通信并传输控制信号。基于C在单片机和WiFi模块上实现建立通信、分析信号与响应控制功能。}
%       \end{cvitems}
%     }
% %---------------------------------------------------------
\end{cventries}
