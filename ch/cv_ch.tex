%!TEX TS-program = xelatex
%!TEX encoding = UTF-8 Unicode
%----------------------------------------------------------------
% CONFIGURATIONS
%----------------------------------------------------------------
% A4纸张大小默认情况下,'letterpaper'使用信纸
\documentclass[11pt, a4paper,AutoFakeBold]{awesome-cv}
% \documentclass[a4paper]{article}
\usepackage[UTF8]{ctex}

% 使用geometry配置页边距
\geometry{left=1.4cm, top=.8cm, right=1.4cm, bottom=1.8cm, footskip=.5cm}
 
% 指定包含字体的位置
% \fontdir[fonts/]

% 高光颜色
% Awesome Colors: awesome-emerald, awesome-skyblue, awesome-red, awesome-pink, awesome-orange
%                 awesome-nephritis, awesome-concrete, awesome-darknight
% \colorlet{awesome}{awesome-red}
% 如果要指定自己的颜色,请取消注释
% \definecolor{awesome}{HTML}{CA63A8}

% 文本的颜色
% 如果要指定自己的颜色,请取消注释
% \definecolor{darktext}{HTML}{414141}
% \definecolor{text}{HTML}{333333}
% \definecolor{graytext}{HTML}{5D5D5D}
% \definecolor{lighttext}{HTML}{999999}

% Set false if you don't want to highlight section with awesome color
% \setbool{acvSectionColorHighlight}{true}

% 如果要将社交信息分隔符从“|”更改为其他内容
\renewcommand{\acvHeaderSocialSep}{\quad\textbar\quad}


%----------------------------------------------------------------
%	如果要将社交信息分隔符从“|”更改为其他内容
%	如果不需要,请注释下面的任何一行
%----------------------------------------------------------------
% 可用选项: circle|rectangle,edge/noedge,left/right
\photo[circle,edge,left]{./examples/profile.jpg}
\name{纪宇}{}
% \position{目标岗位{\enskip\cdotp\enskip}EEDP}
% \position{目标岗位{\enskip\cdotp\enskip}算法工程师}

\address{中共党员, 1999年2月生于山东省德州市\\山东大学控制科学与工程专业硕士}


\mobile{195 1018 4095}
\email{jiyu@mail.sdu.edu.cn}
% \homepage{www.posquit0.com}
\github{migra1315}
% \linkedin{posquit0}
% \gitlab{gitlab-id}
% \stackoverflow{SO-id}{SO-name}
% \twitter{@twit}
% \skype{skype-id}
% \reddit{reddit-id}
% \medium{madium-id}
% \googlescholar{googlescholar-id}{name-to-display}
%% \firstname and \lastname will be used
% \googlescholar{googlescholar-id}{}
% \extrainfo{extra informations}

% \quote{``成为你想在世界上看到的改变。"}


%----------------------------------------------------------------
\begin{document}

% 打印具有以上个人信息的标题
% 提供可选参数以更改对齐方式(C: center, L: left, R: right)
\makecvheader

% 打印带有3个参数的页脚(<left>, <center>, <right>)
% 如果不需要,请将其中任何一项留空
\makecvfooter
  {}
  {纪宇~~~·~~~求职简历}
  {\thepage}



%----------------------------------------------------------------
%	CV 目录
%	每个部分是单独导入的,依次打开每个文件修改内容
%----------------------------------------------------------------
\cvsection{教育经历}


%----------------------------------------------------------------
%	CONTENT
%----------------------------------------------------------------
\begin{cventries}

%---------------------------------------------------------
  \cventry
    {推荐免试, 现控制科学与工程硕士三年级在读。预计2023年6月毕业} % Degree
    {硕士研究生(山东大学, 211、985、双一流建设高校)} % Institution
    {山东济南} % Location
    {2020年9月至今} % Date(s)
    {
      \begin{cvitems} % Description(s) bullet points
        \item {研究方向为面向手术导航的\textbf{医学图像配准技术}}
        \item {研究的主要实现方法为深度学习, 对CNN、GAN、Transformer等模型架构熟悉}
      \end{cvitems}
    }
    \cventry
    {电子信息工程学士/会计学辅修} % Degree
    {本科(中国矿业大学, 211、双一流建设高校)} % Institution
    {江苏徐州} % Location
    {2016年9月-2020年6月} % Date(s)
    {
      \begin{cvitems} % Description(s) bullet points
         \item {\textbf{GPA: }4.06/5}
         \item {\textbf{Rank: }24/254}
      \end{cvitems}
    }
    

%---------------------------------------------------------
\end{cventries}

%----------------------------------------------------------------
%	SECTION TITLE
%----------------------------------------------------------------
\cvsection{技能}


%----------------------------------------------------------------
%	CONTENT
%----------------------------------------------------------------
\begin{cvskills}

%---------------------------------------------------------
  \cvskill
    {编程技能} % 类别
    {熟练掌握Python、Linux, 熟悉C/C++, 了解Java、git, 有Pytorch和Qt项目经历} % 技能

%---------------------------------------------------------
  \cvskill
  {深度学习} % 类别
  {熟练使用Pytorch, 掌握TensorFlow; 对VoxelMorph、CycleGAN、MUNIT等模型架构了解深入} % 技能

% %---------------------------------------------------------  
%   \cvskill
%   {办公软件} % 类别
%   {熟练使用Word、Excel、PowerPoint、Visio、\LaTeX、飞书 等办公及协作软件} % 技能

%---------------------------------------------------------
  \cvskill
    {外语} % 类别
    {英语CET-6通过, 能熟练阅读外文论文与技术文档} % 技能

%---------------------------------------------------------
\end{cvskills}

\cvsection{项目经历}


%----------------------------------------------------------------
%	内容
%----------------------------------------------------------------
\begin{cventries}
%---------------------------------------------------------
  \cventry
    {课题“动态多模影像融合的术中可视化导引技术研究”, 项目组骨干成员}% 职位名称
    {国家重点研发计划“粒子植入近距离放疗微创机器人系统研制及验证”}  % 组织
    {山东大学} % 位置
    {2020年4月至今} % 日期(s)
    {      
      \begin{cvitems} % 任务/职责描述
        \item{\textbf{项目描述: }本项目面向穿刺手术过程中对影像动态配准以及术中导引作业中可视化引导需求, 
        开展基于动态多模态影像数据融合的动态配准与空间目标实时定位、跟踪技术的研究。}
        \item{\textbf{承担工作: }医学图像配准算法开发;手术导航系统搭建与机器人系统集成;项目管理工作}
      \end{cvitems}
    }
  \cventry
    {发表会议论文4篇, 申请专利1项。} % Job title
    {「科研」基于深度学习的医学图像配准算法研究} % Organization
    {} % Location
    {} % Date(s)
    {
      \begin{cvitems}
        \item \textbf{基于时空特征的肺部4D CT配准算法}
        \begin{itemize}[leftmargin=*,topsep=5 pt, partopsep=5 pt]
          \item[$\ast$] 提出一种时空特征融合配准算法, 以弥补现有配准策略中时序运动信息的缺位。
          \item[$\ast$] 设计Conv-LSTM构成的时序建模模块作为 UNet 的 bottleneck 层, 用以处理图像序列的运动特征。
          \item[$\ast$] 构建Dual-stream配准流平衡网络完成对周期运动数据的处理。
          \item[$\ast$] 在Dirlab数据集上较其他算法取得显著提升, 相关工作发表在 BioCAS 2022 (in oral)。
        \end{itemize}
        \item \textbf{多标准融合的多模态医学影像配准算法}
        \begin{itemize}[leftmargin=*,topsep=5 pt, partopsep=5 pt]
          \item[$\ast$] 提出了一种结合跨模态相似度量和图像结构编码度量的多标准配准策略, 以更全局的角度优化网络。
          \item[$\ast$] 通过非成对图像训练图像翻译与解纠缠网络提取 (模态无关) 结构编码, 作为多模态影像相似性度量。
          \item[$\ast$] 设计多标准融合策略, 结合跨模态相似度量和图像结构编码度量以优化网络。
          \item[$\ast$] 优于现有基于图像翻译方法及单标准方法的配准结果。相关工作发表在 ISBI 2022。
        \end{itemize}
      \end{cvitems}
    }
    
  \cventry
    {合作完成。主要负责系统搭建、导航模块开发和GUI设计工作}% 职位名称
    {「项目」手术机器人系统搭建及可视化界面开发}  % 组织
    {} % 位置
    {} % 日期(s)
    {      
      \begin{cvitems} % 任务/职责描述
        \item{以NDI作为场景相机, UR5作为执行机构, 使用Qt设计GUI搭建手术导航与执行系统。}
        \item{规范导航系统功能, 实现影像、场景相机以及执行机构标定流程。}
        \item{搭建相机、机械臂以及HoloLens间的TCP/IP通信, 完成追踪与引导数据的传输交流。}
        \item{设计GUI, 满足手术注册、规划显示、器官及穿刺执行机构动态展示以及执行控制功能。}
      \end{cvitems}
    }

    \cventry
    {负责课题间联系协调、项目文书撰写整合以及系统联调}% 职位名称
    {「项目管理」项目进程组织管理}  % 组织
    {} % 位置
    {} % 日期(s)
    {      
      \begin{cvitems} % 任务/职责描述
        \item{撰写, 协调并整合科技报告、项目申报书。}
        \item{负责与项目其他单位协调集成穿刺机器人系统, 主要负责与手术规划系统和穿刺执行系统的对接和协调。}
        \item{对穿刺手术机器人的规划、导航、执行以及临床操作均有一定的认识了解。}
      \end{cvitems}
    }
%---------------------------------------------------------
  % \cventry
  %   {学院优秀毕业设计} % Job title
  %   {基于互信息的医学图像配准算法研究,毕业设计} % Organization
  %   {中国矿业大学} % Location
  %   {2019年12月-2020年6月} % Date(s)
  %   {
  %     \begin{cvitems} % Description(s) of tasks/responsibilities
  %       \item {项目描述:本项目是毕业设计,取得学院优秀毕业设计。针对多模态医学图像配准任务对精度和速度的需求,设计基于互信息MI的多模态医学图像配准算法研究,
  %       设计图像形态学处理、多分辨率等策略提高配准速度与精度。}
  %       \item {项目职责:使用MATLAB实现对肺部医学图像的形态学预处理,完成肺部特征的提取与增强。基于C++,使用OpenVC设计以互信息为相似性度量的多分辨率医学图像配准算法,使用Powell优化算法驱动图像变形,在BrainWeb数据集上验证配准精度可达到亚像素级别。}
  %     \end{cvitems}
  %   }

  %  %--------------------------------------------------------- 
%   \cventry
%     {电子设计} % Job title
%     {基于WiFi的智能家居系统设计,项目组长} % Organization
%     {中国矿业大学} % Location
%     {2019年5月-2019年6月} % Date(s)
%     {
%       \begin{cvitems} % Description(s) of tasks/responsibilities
%         \item {项目描述:本项目是电子设计。设计实现手机端APP与设备端单片机通过WiFi实现通信的智能家居系统。}
%         \item {项目职责:使用Java实现Android端APP设计,完成对手机传感器等信息的采集、分析,通过WiFi与受控端建立通信并传输控制信号。基于C在单片机和WiFi模块上实现建立通信、分析信号与响应控制功能。}
%       \end{cvitems}
%     }
% %---------------------------------------------------------
\end{cventries}

% %-------------------------------------------------------------------------------
%	SECTION TITLE
%-------------------------------------------------------------------------------
\cvsection{Program Committees}


%-------------------------------------------------------------------------------
%	CONTENT
%-------------------------------------------------------------------------------
\begin{cvhonors}

%---------------------------------------------------------
  \cvhonor
    {Problem Writer} % Position
    {2016 CODEGATE Hacking Competition World Final} % Committee
    {S.Korea} % Location
    {2016} % Date(s)

%---------------------------------------------------------
  \cvhonor
    {Organizer \& Co-director} % Position
    {1st POSTECH Hackathon} % Committee
    {S.Korea} % Location
    {2013} % Date(s)

%---------------------------------------------------------
  \cvhonor
    {Staff} % Position
    {7th Hacking Camp} % Committee
    {S.Korea} % Location
    {2012} % Date(s)

%---------------------------------------------------------
  \cvhonor
    {Problem Writer} % Position
    {1st Hoseo University Teenager Hacking Competition} % Committee
    {S.Korea} % Location
    {2012} % Date(s)

%---------------------------------------------------------
  \cvhonor
    {Staff \& Problem Writer} % Position
    {JFF(Just for Fun) Hacking Competition} % Committee
    {S.Korea} % Location
    {2012} % Date(s)

%---------------------------------------------------------
\end{cvhonors}

\newpage
%-------------------------------------------------------------------------------
%	SECTION TITLE
%-------------------------------------------------------------------------------
\cvsection{Honors \& Awards}


%-------------------------------------------------------------------------------
%	SUBSECTION TITLE
%-------------------------------------------------------------------------------
\cvsubsection{International}


%-------------------------------------------------------------------------------
%	CONTENT
%-------------------------------------------------------------------------------
\begin{cvhonors}

%---------------------------------------------------------
  \cvhonor
    {Finalist} % Award
    {DEFCON 26th CTF Hacking Competition World Final} % Event
    {Las Vegas, U.S.A} % Location
    {2018} % Date(s)

%---------------------------------------------------------
  \cvhonor
    {Finalist} % Award
    {DEFCON 25th CTF Hacking Competition World Final} % Event
    {Las Vegas, U.S.A} % Location
    {2017} % Date(s)

%---------------------------------------------------------
  \cvhonor
    {Finalist} % Award
    {DEFCON 22nd CTF Hacking Competition World Final} % Event
    {Las Vegas, U.S.A} % Location
    {2014} % Date(s)

%---------------------------------------------------------
  \cvhonor
    {Finalist} % Award
    {DEFCON 21st CTF Hacking Competition World Final} % Event
    {Las Vegas, U.S.A} % Location
    {2013} % Date(s)

%---------------------------------------------------------
  \cvhonor
    {Finalist} % Award
    {DEFCON 19th CTF Hacking Competition World Final} % Event
    {Las Vegas, U.S.A} % Location
    {2011} % Date(s)

%---------------------------------------------------------
\end{cvhonors}


%-------------------------------------------------------------------------------
%	SUBSECTION TITLE
%-------------------------------------------------------------------------------
\cvsubsection{Domestic}


%-------------------------------------------------------------------------------
%	CONTENT
%-------------------------------------------------------------------------------
\begin{cvhonors}

%---------------------------------------------------------
  \cvhonor
    {3rd Place} % Award
    {WITHCON Hacking Competition Final} % Event
    {Seoul, S.Korea} % Location
    {2015} % Date(s)

%---------------------------------------------------------
  \cvhonor
    {Silver Prize} % Award
    {KISA HDCON Hacking Competition Final} % Event
    {Seoul, S.Korea} % Location
    {2017} % Date(s)

%---------------------------------------------------------
  \cvhonor
    {Silver Prize} % Award
    {KISA HDCON Hacking Competition Final} % Event
    {Seoul, S.Korea} % Location
    {2013} % Date(s)

%---------------------------------------------------------
\end{cvhonors}

% %-------------------------------------------------------------------------------
%	SECTION TITLE
%-------------------------------------------------------------------------------
\cvsection{Extracurricular Activity}


%-------------------------------------------------------------------------------
%	CONTENT
%-------------------------------------------------------------------------------
\begin{cventries}

%---------------------------------------------------------
  \cventry
    {Core Member \& President at 2013} % Affiliation/role
    {PoApper (Developers' Network of POSTECH)} % Organization/group
    {Pohang, S.Korea} % Location
    {Jun. 2010 - Jun. 2017} % Date(s)
    {
      \begin{cvitems} % Description(s) of experience/contributions/knowledge
        \item {Reformed the society focusing on software engineering and building network on and off campus.}
        \item {Proposed various marketing and network activities to raise awareness.}
      \end{cvitems}
    }

%---------------------------------------------------------
  \cventry
    {Member} % Affiliation/role
    {PLUS (Laboratory for UNIX Security in POSTECH)} % Organization/group
    {Pohang, S.Korea} % Location
    {Sep. 2010 - Oct. 2011} % Date(s)
    {
      \begin{cvitems} % Description(s) of experience/contributions/knowledge
        \item {Gained expertise in hacking \& security areas, especially about internal of operating system based on UNIX and several exploit techniques.}
        \item {Participated on several hacking competition and won a good award.}
        \item {Conducted periodic security checks on overall IT system as a member of POSTECH CERT.}
        \item {Conducted penetration testing commissioned by national agency and corporation.}
      \end{cvitems}
    }

%---------------------------------------------------------
\end{cventries}

% %-------------------------------------------------------------------------------
%	SECTION TITLE
%-------------------------------------------------------------------------------
\cvsection{Presentation}


%-------------------------------------------------------------------------------
%	CONTENT
%-------------------------------------------------------------------------------
\begin{cventries}

%---------------------------------------------------------
  \cventry
    {Presenter for <Hosting Web Application for Free utilizing GitHub, Netlify and CloudFlare>} % Role
    {DevFest Seoul by Google Developer Group Korea} % Event
    {Seoul, S.Korea} % Location
    {Nov. 2017} % Date(s)
    {
      \begin{cvitems} % Description(s)
        \item {Introduced the history of web technology and the JAM stack which is for the modern web application development.}
        \item {Introduced how to freely host the web application with high performance utilizing global CDN services.}
      \end{cvitems}
    }

%---------------------------------------------------------
  \cventry
    {Presenter for <DEFCON 20th : The way to go to Las Vegas>} % Role
    {6th CodeEngn (Reverse Engineering Conference)} % Event
    {Seoul, S.Korea} % Location
    {Jul. 2012} % Date(s)
    {
      \begin{cvitems} % Description(s)
        \item {Introduced CTF(Capture the Flag) hacking competition and advanced techniques and strategy for CTF}
      \end{cvitems}
    }

%---------------------------------------------------------
  \cventry
    {Presenter for <Metasploit 101>} % Role
    {6th Hacking Camp - S.Korea} % Event
    {S.Korea} % Location
    {Sep. 2012} % Date(s)
    {
      \begin{cvitems} % Description(s)
        \item {Introduced basic procedure for penetration testing and how to use Metasploit}
      \end{cvitems}
    }

%---------------------------------------------------------
\end{cventries}

% %----------------------------------------------------------------
%	SECTION TITLE
%----------------------------------------------------------------
\cvsection{Pub List}


%----------------------------------------------------------------
%	CONTENT
%----------------------------------------------------------------
\begin{cventries}

--------------------------------------------------------
%   \cventry
%     {创始人兼作家} % Role
%     {初学者指南} % Title
%     {Facebook页面} % Location
%     {2015年1月至今} % Date(s)
%     {
%       \begin{cvitems} % Description(s)
%         \item {为韩国的开发者起草关于IT技术和创业问题的每日新闻。}
%       \end{cvitems}
%     }

%---------------------------------------------------------
\end{cventries}

\cvsection{Pub List}

[1] \textbf{Y. Ji}, Z. Zhu and Y. Wei, "A One-Shot Lung 4D-CT Image Registration Method with Temporal-Spatial Features," 2022 IEEE Biomedical Circuits and Systems Conference (BioCAS), 2022「Oral」

[2] \textbf{Y. Ji}, Z. Zhu and Y. Wei, "Fusion-Based Multimodal Medical Image Registration Combining Inter-Modality Metric and Disentanglement," 2022 IEEE 19th International Symposium on Biomedical Imaging (ISBI), 2022

[3] 魏莹, \textbf{纪宇}, "基于多标准融合的多模态医学图像配准方法及系统,"202210095355.8(实质审查)

[4] Z. Zhu, \textbf{Y. Ji} and Y. Wei, "Multi-Resolution Medical Image Registration with Dynamic Convolution," 2022 IEEE Biomedical Circuits and Systems Conference (BioCAS), 2022

[5] Z. Zhu, \textbf{Y. Ji} and Y. Wei, "Lung CT image registration based on end-to-end unsupervised learning," 2021 6th International Conference on Communication, Image and Signal Processing (CCISP), 2021
%----------------------------------------------------------------
\end{document}
