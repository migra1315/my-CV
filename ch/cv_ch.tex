%!TEX TS-program = xelatex
%!TEX encoding = UTF-8 Unicode
%----------------------------------------------------------------
% CONFIGURATIONS
%----------------------------------------------------------------
% A4纸张大小默认情况下,'letterpaper'使用信纸
\documentclass[11pt, a4paper,AutoFakeBold]{awesome-cv}
% \documentclass[a4paper]{article}
\usepackage[UTF8]{ctex}

% 使用geometry配置页边距
\geometry{left=1.4cm, top=.8cm, right=1.4cm, bottom=1.8cm, footskip=.5cm}
 
% 指定包含字体的位置
% \fontdir[fonts/]

% 高光颜色
% Awesome Colors: awesome-emerald, awesome-skyblue, awesome-red, awesome-pink, awesome-orange
%                 awesome-nephritis, awesome-concrete, awesome-darknight
% \colorlet{awesome}{awesome-red}
% 如果要指定自己的颜色,请取消注释
% \definecolor{awesome}{HTML}{CA63A8}

% 文本的颜色
% 如果要指定自己的颜色,请取消注释
% \definecolor{darktext}{HTML}{414141}
% \definecolor{text}{HTML}{333333}
% \definecolor{graytext}{HTML}{5D5D5D}
% \definecolor{lighttext}{HTML}{999999}

% Set false if you don't want to highlight section with awesome color
% \setbool{acvSectionColorHighlight}{true}

% 如果要将社交信息分隔符从“|”更改为其他内容
\renewcommand{\acvHeaderSocialSep}{\quad\textbar\quad}


%----------------------------------------------------------------
%	如果要将社交信息分隔符从“|”更改为其他内容
%	如果不需要,请注释下面的任何一行
%----------------------------------------------------------------
% 可用选项: circle|rectangle,edge/noedge,left/right
\photo[circle,edge,left]{./examples/profile.jpg}
\name{纪宇}{}
% \position{目标岗位{\enskip\cdotp\enskip}EEDP}
% \position{目标岗位{\enskip\cdotp\enskip}算法工程师}

\address{中共党员, 1999年2月生于山东省德州市\\山东大学控制科学与工程专业硕士}


\mobile{195 1018 4095}
\email{jiyu@mail.sdu.edu.cn}
% \homepage{www.posquit0.com}
\github{migra1315}
% \linkedin{posquit0}
% \gitlab{gitlab-id}
% \stackoverflow{SO-id}{SO-name}
% \twitter{@twit}
% \skype{skype-id}
% \reddit{reddit-id}
% \medium{madium-id}
% \googlescholar{googlescholar-id}{name-to-display}
%% \firstname and \lastname will be used
% \googlescholar{googlescholar-id}{}
% \extrainfo{extra informations}

% \quote{``成为你想在世界上看到的改变。"}


%----------------------------------------------------------------
\begin{document}

% 打印具有以上个人信息的标题
% 提供可选参数以更改对齐方式(C: center, L: left, R: right)
\makecvheader

% 打印带有3个参数的页脚(<left>, <center>, <right>)
% 如果不需要,请将其中任何一项留空
\makecvfooter
  {}
  {纪宇~~~·~~~求职简历}
  {\thepage}



%----------------------------------------------------------------
%	CV 目录
%	每个部分是单独导入的,依次打开每个文件修改内容
%----------------------------------------------------------------
%-------------------------------------------------------------------------------
%	SECTION TITLE
%-------------------------------------------------------------------------------
\cvsection{Education}


%-------------------------------------------------------------------------------
%	CONTENT
%-------------------------------------------------------------------------------
\begin{cventries}

  %---------------------------------------------------------
  \cventry
    {M.E. candidate in Control Science and Engineering} % Degree
    {SDU ( Shandong University, 985, 211)} % Institution
    {Jinan, Shandong} % Location
    {Sep. 2020 - Now} % Date(s)
    {
      \begin{cvitems} % Description(s) bullet points
        \item {Researched on Medical Image Registration and Navigation of Puncture Surgery Robot}
        \item {Implemented related works using deep Learning methods}
      \end{cvitems}
    }

%---------------------------------------------------------
  \cventry
    {B.E. in Electronic Information Engineering} % Degree
    {CUMT ( China University of Mining and Technology, 211)} % Institution
    {Xuzhou, Jiangsu} % Location
    {Sep. 2016 - June. 2020} % Date(s)
    {
      \begin{cvitems} % Description(s) bullet points
        \item {GPA: 4.06/5}
        \item {RANK: 24/254}
      \end{cvitems}
    }

%---------------------------------------------------------
\end{cventries}

%-------------------------------------------------------------------------------
%	SECTION TITLE
%-------------------------------------------------------------------------------
\cvsection{Skills}


%-------------------------------------------------------------------------------
%	CONTENT
%-------------------------------------------------------------------------------
\begin{cvskills}

% %---------------------------------------------------------
%   \cvskill
%     {DevOps} % Category
%     {AWS, Docker, Kubernetes, Rancher, Vagrant, Packer, Terraform, Jenkins, CircleCI} % Skills

% %---------------------------------------------------------
%   \cvskill
%     {Back-end} % Category
%     {Koa, Express, Django, REST API} % Skills

% %---------------------------------------------------------
%   \cvskill
%     {Front-end} % Category
%     {Hugo, Redux, React, HTML5, LESS, SASS} % Skills

%---------------------------------------------------------
  \cvskill
    {Programming} % Category
    {Python, C/C++, LaTeX, git; Experience in Python and Qt development} % Skills
%---------------------------------------------------------
  \cvskill
  {Deep Learning} % Category
  {Pytorch, Tensorflow; Familiar with registration, generation and other CV algorithms} % Skills

%---------------------------------------------------------
  \cvskill
    {Languages} % Category
    {English; Proficient in reading foreign papers and technical documents} % Skills

%---------------------------------------------------------
\end{cvskills}

%-------------------------------------------------------------------------------
%	SECTION TITLE
%-------------------------------------------------------------------------------

\cvsection{Experience}

%-------------------------------------------------------------------------------
%	CONTENT
%-------------------------------------------------------------------------------
\begin{cventries}
%---------------------------------------------------------
  \cventry
    {Development and validation of a minimally invasive robot system for particle implantation brachytherapy} % Job title
    {National key R\&D Development} % Organization
    {Shandong University} % Location
    {Apr. 2020 - Now} % Date(s)
    {
      \begin{cvitems} % Description(s) of tasks/responsibilities
        \item {Undertook the Research topic "Intraoperative Visual Guidance of Dynamic Multi-modality Fusion" as a key member of the project team.}
        % \item {Aiming to meet the requirements of image registration and visualization during intraoprative guidance, 
        % this topic carries out research dynamic registration, real-time positioning and tracking of spatial targets.}
        \item {Developed several medical image deep learning-based registration algorithm applied for CT and multi-modality MR images.}
        \item {Implemented a navigation and execution robot system for surgical which performed image guided puncture operation. }
        \item {Actualized the project management such as system integration commissoning, project declaration cooperation and demand scheduling.}
      \end{cvitems}
    }
%---------------------------------------------------------
  \cventry
    {Published 4 conference papers and applied for 1 patent. } % Job title
    {$\lceil$ 1 $\rfloor$ Research on medical image registration based on deep learning} % Organization
    {} % Location
    {} % Date(s)
    {
      \textbf{Registration Methods for 4D CT of Lung Based on Temporal-Spatial Features.}
      \begin{itemize}[leftmargin=*]
        \item Developed a Temporal-Spatial Features fusion registration method to make up the absence of temporal characteristics in existing methods.
        \item Designed a novel temporal modeling module inserted into the bottleneck layer of UNet to process motion information .
        \item Proposed a Dual-stream registration Pipeline to fit the relationship between data and network.
        \item Achieved significant improvement results. Realated works published on BioCAS 2022 (in oral).
      \end{itemize}
      \textbf{Multimodal Medical Image Registration based on Multi Metric Fusion}
      \begin{itemize}[leftmargin=*]
        \item Developed a multi metric registrion method combined intra-modality similarity and structure code metric to optimize network in a more global perspective.
        \item Extracted structure information from multimodal images by image dsentanglement as multimodal similarity metric.
        \item Designed fusion strategy for dual-metric registration model optimization with both intra-modality similarity and structure code.
        \item Outperformed on translation-based and single metric registration methods. Realated works published on ISBI 2022.
      \end{itemize}
    }
%---------------------------------------------------------
  \cventry
    {Development of Robot System, navigation, and related software  } % Job title
    {$\lceil$ 2 $\rfloor$ Development of surgical robot system with visual guidance} % Organization
    {} % Location
    {} % Date(s)
    {
      \begin{cvitems} % Description(s) of tasks/responsibilities
        \item Build the surgical navigation and execution system which takes NDI as the scene camera, UR5 as the executive mechanism, and Qt to design GUI.
        \item Standardized the navigation system functions, completed the calibration process of image, scene camera and robot.
        \item Realized TCP/IP communication with equipment for the exchange of tracking and guidance data
        \item Designed GUI for the functions of surgical navigation, dynamic display, and robot execution control.
      \end{cvitems}
    }
%---------------------------------------------------------
\cventry
{Coordination among topics, project document writing and integration} % Job title
{$\lceil$ 3 $\rfloor$ Project process organization and management} % Organization
{} % Location
{} % Date(s)
{
  \begin{cvitems} % Description(s) of tasks/responsibilities
    \item Wrote and integrated project declaration and sci-tech report.
    \item Coordinated with other units of the project to integrate the puncture robot system.
    \item Familiar with the planning, navigation, execution and clinical operation of the puncture surgery robot.
  \end{cvitems}
}
%---------------------------------------------------------
  % \cventry
  %   {Researcher} % Job title
  %   {Undergraduate Research, Machine Learning Lab(Prof. Seungjin Choi)} % Organization
  %   {Pohang, S.Korea} % Location
  %   {Mar. 2016 - Exp. Jun. 2017} % Date(s)
  %   {
  %     \begin{cvitems} % Description(s) of tasks/responsibilities
  %       \item {Researched classification algorithms(SVM, CNN) to improve accuracy of human exercise recognition with wearable device.}
  %       \item {Developed two TIZEN applications to collect sample data set and to recognize user exercise on SAMSUNG Gear S.}
  %     \end{cvitems}
  %   }
%---------------------------------------------------------
\end{cventries}

% \cvsection{学生工作}


%----------------------------------------------------------------
%	CONTENT
%----------------------------------------------------------------
\begin{cvhonors}

%---------------------------------------------------------
  \cvhonor
    {中国矿业大学校务参事/年级长/班长} % Position
    {} % Committee
    {中国矿业大学} % Location
    {2016-2020} % Date(s)
%---------------------------------------------------------
%   \cvhonor
%     {中国矿业大学信控学院} % Position
%     {} % Committee
%     {中国矿业大学信控学院} % Location
%     {2016-2019} % Date(s)
% %---------------------------------------------------------
\end{cvhonors}

\newpage
%-------------------------------------------------------------------------------
%	SECTION TITLE
%-------------------------------------------------------------------------------
\cvsection{School Committees \& Honors}


%-------------------------------------------------------------------------------
%	SUBSECTION TITLE
%-------------------------------------------------------------------------------
\cvsubsection{School Committees}

%-------------------------------------------------------------------------------
%	CONTENT
%-------------------------------------------------------------------------------
\begin{cvhonors}

%---------------------------------------------------------
  \cvhonor
    {} % Award
    {Class Monitor/Student Union President/Student Assistant of University President} % Event
    {CUMT} % Location
    {2016-2020} % Date(s)
\end{cvhonors}

%-------------------------------------------------------------------------------
%	SUBSECTION TITLE
%-------------------------------------------------------------------------------
\cvsubsection{Honors}


%-------------------------------------------------------------------------------
%	CONTENT
%-------------------------------------------------------------------------------
\begin{cvhonors}
%---------------------------------------------------------
  \cvhonor
    {Scholarship for Outstanding Students/Academic Scholarship} % Award
    {less than 30\% in SDU} % Event
    {SDU} % Location
    {2020/2021} % Date(s)
%---------------------------------------------------------
  \cvhonor
    {Excellent graduates} % Award
    {less than 5\% in CUMT} % Event
    {CUMT} % Location
    {2020} % Date(s)
%---------------------------------------------------------
  \cvhonor
    {Excellent student cadres} % Award
    {less than 1\textperthousand ~ in Jiangsu Province} % Event
    {Jiangsu Education Department} % Location
    {2019} % Date(s)
%---------------------------------------------------------
  \cvhonor
    {Excellent student cadres} % Award
    {less than 5\% in CUMT} % Event
    {CUMT} % Location
    {2018/2019} % Date(s)
%---------------------------------------------------------
\end{cvhonors}

% \input{cv/extracurricular.tex}
% \cvsection{其他表现}


%----------------------------------------------------------------
%	CONTENT
%----------------------------------------------------------------
\begin{cventries}

%---------------------------------------------------------
\cventry
  {} % Role
  {效率工具达人/强自我驱动力/文档整理爱好者} % Event
  {} % Location
  {} % Date(s)
  {\begin{cvitems} % Description(s)
    \item {推动实验室办公APP迁移到飞书, 有效提高工作协同效率; 在实验室内部推广Obsidian、坚果云、flomo、EveryThing等效率工具}
    \item {完成实验室服务器的配置与文档整理, 完成影像处理、导航系统等文档整理}
    \item {本科期间组织课程资料整合与分享,牵头完成学院年级内、年级间传帮带体系建设与完善}
    \item {本科期间调研学校学风建设, 发放问卷600份+, 走访教师、学生40人次+, 整理形成3份相关建议与提案上交学校, 并得到落实}
  \end{cvitems}
  }
%---------------------------------------------------------
\end{cventries}

% \input{cv/writing.tex}
\cvsection{Pub List}

[1] \textbf{Y. Ji}, Z. Zhu and Y. Wei, "A One-Shot Lung 4D-CT Image Registration Method with Temporal-Spatial Features," 2022 IEEE Biomedical Circuits and Systems Conference (BioCAS), 2022「Oral」

[2] \textbf{Y. Ji}, Z. Zhu and Y. Wei, "Fusion-Based Multimodal Medical Image Registration Combining Inter-Modality Metric and Disentanglement," 2022 IEEE 19th International Symposium on Biomedical Imaging (ISBI), 2022

[3] 魏莹, \textbf{纪宇}, "基于多标准融合的多模态医学图像配准方法及系统,"202210095355.8(实质审查)

[4] Z. Zhu, \textbf{Y. Ji} and Y. Wei, "Multi-Resolution Medical Image Registration with Dynamic Convolution," 2022 IEEE Biomedical Circuits and Systems Conference (BioCAS), 2022

[5] Z. Zhu, \textbf{Y. Ji} and Y. Wei, "Lung CT image registration based on end-to-end unsupervised learning," 2021 6th International Conference on Communication, Image and Signal Processing (CCISP), 2021
%----------------------------------------------------------------
\end{document}
